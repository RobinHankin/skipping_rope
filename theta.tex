\documentclass{article}
\usepackage{amsmath}
\usepackage[round]{natbib}

\begin{document}

\newcommand{\jj}{\sqrt{1+\left(y'\right)^2}}
\section*{Abstract}
In this short article I discuss skipping ropes



\section{Introduction}

The dynamics of skipping ropes is an everyday yet fascinating topic.
Here I analyse the dynamics of a slowly rotating skipper.  This
document use sarc length and angle $(s,\theta)$, solving for
$\theta=\theta(s)$.  File {\tt skipping.txt} uses Cartesian
coordinates $(x,y)$, solving for $y=y(x)$.

\citep{mohazzabi1999}

{\bf\large In the following, a dash means differentiation with respect
  to arc length}.


With arclength $s$ inclined at $\theta$ to the horizontal and
Cartesian $(x,y)$ coordinates and [possibly nonconservative] force
field

\begin{equation}
  \begin{bmatrix}F\\G\end{bmatrix}
    =
    \begin{bmatrix}F(x,y)\\G(x,y)\end{bmatrix}
\end{equation}
and tensions $T,U$ [formerly $T_x,T_y$] we get

\begin{eqnarray}
T'=  \frac{dT}{ds}=-F\label{Tx}\\
U'=  \frac{dU}{ds}=-G\label{Ty}
  \end{eqnarray}

($F$ is upward and $G$ to the right).

\begin{equation}
  \frac{U}{T}=\frac{dy}{dx}=\tan\theta.
\end{equation}


\begin{equation}
U=T\tan\theta\label{uttantheta}
\end{equation}

For any $X=X(s)$:
\begin{equation}
X'=\frac{dX}{ds}=\frac{dx}{ds}\frac{dX}{dx}+\frac{dy}{ds}\frac{dX}{dy}=X_x\cos\theta + X_y\sin\theta
\end{equation}
(here $X_x=dX/dx$ and $X_y=dX/dy$).  Differentiating equation
\ref{uttantheta} with respect to $s$ gives:

\begin{equation}
  -G=dU/ds=U'=(T\tan\theta)'=T'\tan\theta + T\theta'\sec^2\theta
\end{equation}


\begin{equation}
  T\theta'\sec^2\theta=G+F\tan\theta
\end{equation}

Rearranging:

\begin{equation}
T=\frac{-G-F\tan\theta}{\theta'\sec^2\theta}
\end{equation}

Differentiate once more with respect to $s$ and note again that
$T'=-F$ to obtain

\begin{equation}
  +F=\frac
{    \theta'\sec^2\theta\left[G'+F'\tan\theta+F\theta'\sec^2\theta\right]-(G+F\tan\theta)(\theta''\sec^2\theta+2(\theta')^2\sec^2\theta\tan\theta)
  }{
    (\theta')^2\sec^4\theta
  }
\end{equation}

where $F'=dF/ds=F_x\cos\theta+F_y\sin\theta$ and
$G'=dG/ds=G_x\cos\theta+G_y\sin\theta$.  This is a second-order
differential equation in $\theta=\theta(s)$.
Cancelling $\sec^2\theta$  gives

\begin{equation}
  F=\frac{\theta'\left[G'+F'\tan\theta+F\theta'\sec^2\theta\right]-(G+F\tan\theta)(\theta''+2(\theta')^2\tan\theta)
  }{
    (\theta')^2\sec^2\theta
  }
\end{equation}


Algebra gives

\begin{equation}
  F(\theta')^2\sec^2\theta + (G+F\tan\theta)(\theta''+2(\theta')^2\tan\theta)
  =
\theta'\left[G'+F'\tan\theta+F\theta'\sec^2\theta\right]
\end{equation}

\begin{eqnarray}
  (G+F\tan\theta)(\theta''+2(\theta')^2\tan\theta)
  &=&
  -F(\theta')^2\sec^2\theta
  +
  \theta'\left[G'+F'\tan\theta+F\theta'\sec^2\theta\right]\\
  &=&    \theta'\left[G'+F'\tan\theta\right] 
\end{eqnarray}

\begin{equation}
  \theta''=
  \frac{
    \theta'\left[G'+F'\tan\theta\right]
  }{
    G+F\tan\theta
  }
  -2(\theta')^2\tan\theta
\end{equation}



\bibliography{skipping}
\bibliographystyle{plainnat}
\end{document}
